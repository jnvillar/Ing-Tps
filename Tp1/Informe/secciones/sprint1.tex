\section{Sprint 1: }

\subsection{User Stories: }

    
    \begin{table}[H]
        \centering
        \begin{tabular}{| p{12.118cm} |}
            \hline
            \multicolumn{1}{|c|}{Tasks Iniciales} \\
            \hline
            1. Discutir Diseño del Modelo Inicial. (2 hs.) \\ \hline
            2. Elegir Lenguaje y Framework. (1 hs.) \\ \hline
            3. Agregar Bares, Usuarios y Categorías al Sistema. (1 hs.) \\ \hline
            4. Implementar el Modelo: Crear clase Bar. Crear clase Usuario. Crear clase Categoria. Crear clase Principal. (12 hs.) \\ \hline
        \end{tabular}
    \end{table}
    
    \begin{table}[H]
        \centering
        \begin{tabular}{| p{1.5cm}| p{10.2cm} |}
            \hline
            \multicolumn{2}{|c|}{User Story: Ciudadano Común - Ver Bares Cercanos} \\
            \hline
            COMO & Ciudadano Común \\ \hline
            QUIERO & Ver los bares cercanos a un punto dado \\ \hline
            PARA & Llegar rápido a un bar que me guste \\ \hline
            \hline
        \end{tabular}
        \begin{tabular}{| p{12.118cm} |}
            \multicolumn{1}{|c|}{Tasks} \\
            \hline
            1. Obtiene la ubicación de un usuario o una dirección que inserta el usuario. (1 hs.) \\ \hline
            2. Calcula las distancias a los distintos Bares y muestra los bares a menos de 400 metros. (1 hs.)\\ \hline
            \hline
            \multicolumn{1}{|c|}{Criterios de Aceptación} \\
            \hline
            1. El usuario, dado un punto, debe poder ver los bares cercanos (menos de 400 mentros) a ese lugar \\ \hline
            2. El usuario debe poder volver a la pantalla anterior en cualquier momento \\ \hline
            Value: 10 \\ \hline
            Effort: 8 \\ \hline
        \end{tabular}
    \end{table}
    
    \strong{Justificación:} El enunciado del sprint pide obtener los bares a menos de 400 metros desde un punto dado. \\
    
    \begin{table}[H]
        \centering
        \begin{tabular}{| p{1.5cm}| p{10.2cm} |}
            \hline
            \multicolumn{2}{|c|}{User Story: Ciudadano Común - Filtrar Bares con Wi-Fi} \\
            \hline
            COMO & Ciudadano Común \\ \hline
            QUIERO & Filtrar los bares que tienen Wi-Fi \\ \hline
            PARA & Usar internet en el bar \\ \hline
            \hline
        \end{tabular}
        \begin{tabular}{| p{12.118cm} |}
            \multicolumn{1}{|c|}{Tasks} \\
            \hline
            1. Hacer el Filtro: Obtener el puntaje sobre la categoría. Ordenar la información. Presentar la información ordenada. (2 hs.)\\ \hline
            \hline
            \multicolumn{1}{|c|}{Criterios de Aceptación} \\
            \hline
            1. El usuario, dada la lista de bares cercanos, se le pregunta si quiere filtrar los bares que tienen Wi-Fi \\ \hline
            2. El usuario debe poder volver a la pantalla anterior en cualquier momento \\ \hline
            3. Si el usuario decide filtrar, se quitaran de la lista los bares sin Wi-Fi \\ \hline
            Value: 8 \\ \hline
            Effort: 2 \\ \hline
        \end{tabular}
    \end{table}
    
    \strong{Justificación:} El enunciado del sprint pide obtener los bares que tengan Wi-Fi. \\
    


%%%%%% de aca en mas son todos comentarios %%%%%%%


\begin{comment}

\begin{itemize}
    \begin{itemize}
        \item Task Inicial - Discutir Diseño del Modelo Inicial
        
        \strong{Hacer:}
        
        
        \item Task Inicial - Elegir Lenguaje y Framework
        
        \strong{Hacer:}
        
        
        \item Task Inicial - Agregar Bares, Usuarios y Categorías al Sistema
        
        \strong{Hacer:}
        
        \begin{itemize}
            \item Crear Instancias de usuarios, bares y categorías.
            \item Llenar con informacion las instancias creadas.
            \item Agregar la instancias al Sistemas.
        \end{itemize}
        
        \item Task Inicial - Implementar el Modelo
        
        \strong{Hacer:}
        
        \begin{itemize}
            \item Crear clase Bar.
            \item Crear clase Usuario.
            \item Crear clase Categoria.
            \item Crear clase Principal.
        \end{itemize}
        
    \end{itemize}

    
    \item Ciudadano Común - Ver Bares Cercanos
    
    COMO Ciudadano Común QUIERO ver los bares cercanos a un punto dado PARA poder llegar rápido a un bar que me guste.
    
    \strong{Criterios de Aceptación:}
    
    \begin{itemize}
        \item 1. El usuario, dado un punto, debe poder ver los bares cercanos (menos de 400 mentros) a ese lugar.
        \item 2. El usuario debe poder volver a la pantalla anterior en cualquier momento.
    \end{itemize}
    
    \strong{Value: 10}
    
    \strong{Effort: 8}
    
    \strong{Justificación:} El enunciado del sprint pide obtener los bares a menos de 400 metros desde un punto dado.

    \strong{Tasks:}

    \begin{itemize}
        
        \item Obtener Ubicación (1hs)
        
        \strong{Hacer:} 
        \begin{itemize}
            \item Obtiene la ubicación de un usuario o una dirección que inserta el usuario.
        \end{itemize}
        
        \item Obtener Distancias (1hs)
        
        \strong{Hacer:} 
        \begin{itemize}
            \item Calcula las distancias a los distintos Bares. 
            \item Muestra los bares a menos de 400 metros.
        \end{itemize}
        
    \end{itemize}
    
    
    \item Ciudadano Común - Filtrar Bares con Wi-Fi
    
    COMO Ciudadano Común QUIERO poder filtrar los bares que tienen Wi-Fi PARA poder usar internet en el bar.
    
    \strong{Criterios de Aceptación:}
    
    \begin{itemize}
        \item 1. El usuario, dada la lista de bares cercanos, se le pregunta si quiere filtrar los bares que tienen Wi-Fi
        \item 2. El usuario debe poder volver a la pantalla anterior en cualquier momento.
        \item 3. Si el usuario decide filtrar, se quitaran de la lista los bares sin Wi-Fi
    \end{itemize}
    
    \strong{Value: 8}
    
    \strong{Effort: 2}
    
    \strong{Justificación:} El enunciado del sprint pide obtener los bares que tengan Wi-Fi.
    
    \strong{Tasks:}
    
    \begin{itemize}
        \item Hacer el Filtro (2hs)
        
        \strong{Hacer:} 
        \begin{itemize}
            \item Obtener el puntaje sobre la categoría.
            \item Ordenar la información.
            \item Presentar la información ordenada.
        \end{itemize}
        
    \end{itemize}
    
\end{comment}


\begin{comment}
    \begin{table}[htb]
        \centering
        \begin{tabular}{| p{2.2cm}| p{4.2cm} |}
            \hline
            \multicolumn{2}{|c|}{Ciudadano Común - Ver Bares Cercanos} \\
            \hline \hline
            COMO & Ciudadano Común \\ \hline
            QUIERO & Ver los bares cercanos a un punto dado \\ \hline
            PARA & Poder llegar rápido a un bar que me guste \\ \hline
        \end{tabular}
    \end{table}
    
    \begin{table}[htb]
        \centering
        \begin{tabular}{| p{6.8cm} |}
            \hline
            Tasks \\
            \hline \hline
            1. Obtiene la ubicación de un usuario o una dirección que inserta el usuario \\ \hline
            2. Calcula las distancias a los distintos Bares y muestra los bares a menos de 400 metros\\ \hline
        \end{tabular}
    \end{table}
    
    \begin{table}[htb]
        \centering
        \begin{tabular}{| p{6.8cm} |}
            \hline
            Criterios de Aceptación \\
            \hline \hline
            1. El usuario, dado un punto, debe poder ver los bares cercanos (menos de 400 mentros) a ese lugar \\ \hline
            2. El usuario debe poder volver a la pantalla anterior en cualquier momento \\ \hline
        \end{tabular}
    \end{table}
\end{comment}

