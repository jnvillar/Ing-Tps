\section{Backlog:}

\subsection{User Stories:}

    \begin{table}[H]
        \centering
        \begin{tabular}{| p{1.5cm}| p{10.2cm} |}
            \hline
            \multicolumn{2}{|c|}{User Story: Admin - Aprobar Bar} \\
            \hline
            COMO & Administrador \\ \hline
            QUIERO & Evaluar la aprobación de un bar \\ \hline
            PARA & Moderar el contenido de la aplicación \\ \hline
            \hline
        \end{tabular}
        \begin{tabular}{| p{12.118cm} |}
            \multicolumn{1}{|c|}{Tasks} \\
            \hline
            1.  \\ \hline
            2.  \\ \hline
            \hline
            \multicolumn{1}{|c|}{Criterios de Aceptación} \\
            \hline
            1. El administrador puede acceder a una lista de bares pendientes para su aprobación. \\ \hline
            2. El administrador puede seleccionar un bar y ver su información. \\ \hline
            3. El administrador puede aceptar o rechazar el bar seleccionado. \\ \hline
            4. Luego el bar, si es aceptado, se agrega a la base de datos de la aplicación. \\ \hline
            5. Sino, se descarta toda su información. \\ \hline
            Value: 1 \\ \hline
            Effort: 1 \\ \hline
        \end{tabular}
    \end{table}
    
    \strong{Justificación:} Luego de que un usuario registrado agregue un bar, un admin debe moderar el contenido, para que se esta forma, la aplicación no se llene de información falsa. \\
    
    \begin{table}[H]
        \centering
        \begin{tabular}{| p{1.5cm}| p{10.2cm} |}
            \hline
            \multicolumn{2}{|c|}{User Story: Admin - Modificar/Eliminar un Bar} \\
            \hline
            COMO & Administrador \\ \hline
            QUIERO & Poder modificar la información de un bar seleccionado e incluso borrarlo \\ \hline
            PARA & Moderar el contenido de la aplicación \\ \hline
            \hline
        \end{tabular}
        \begin{tabular}{| p{12.118cm} |}
            \multicolumn{1}{|c|}{Tasks} \\
            \hline
            1. \\ \hline
            2. \\ \hline
            \hline
            \multicolumn{1}{|c|}{Criterios de Aceptación} \\
            \hline
            1. El Admin debe poder modificar, el nombre y la ubicación del bar seleccionado. \\ \hline
            2. Al Admin se le pedirá que confirme o descarte los cambios \\ \hline
            3. El Admin debe poder eliminar un bar seleccionado. \\ \hline
            4. Al Admin se le pedirá que confirme la eliminación del bar. \\ \hline
            5. Al terminar el accionar del Admin, se guardaran los cambios (en caso de que los haya) en la base de datos. \\ \hline
            Value: 3 \\ \hline
            Effort: 5 \\ \hline
        \end{tabular}
    \end{table}
    
    \strong{Justificación:} Queremos que la posibilidad de modificar/eliminar un bar esté disponible para el administrador, ya que podría ser útil si se ingreso información falsa, o si se desea quitar un bar de la app. \\
    
    \begin{table}[H]
        \centering
        \begin{tabular}{| p{1.5cm}| p{10.2cm} |}
            \hline
            \multicolumn{2}{|c|}{User Story: Ciudadano Resgistrado - Agregar Bar} \\
            \hline
            COMO & Ciudadano Registrado \\ \hline
            QUIERO & Poder agregar un nuevo bar \\ \hline
            PARA & Que aparezca en la aplicación \\ \hline
            \hline
        \end{tabular}
        \begin{tabular}{| p{12.118cm} |}
            \multicolumn{1}{|c|}{Tasks} \\
            \hline
            1. \\ \hline
            2. \\ \hline
            \hline
            \multicolumn{1}{|c|}{Criterios de Aceptación} \\
            \hline
            1. El usuario debe ingresar el nombre y la ubicación del bar. \\ \hline
            2. El usuario debe obligatoriamente, indicar si el bar tiene Wi-Fi o no, y en caso de que lo tenga, calificar el nivel del Wi-Fi. Además debe calificar la cantidad de enchufes del bar. \\ \hline
            3. El usuario puede calificar las otras características que haya en la aplicación. \\ \hline
            4. Al usuario se le muestra un resumen de lo que subió y se le pide una confirmación. \\ \hline
            5. Se le notifica al usuario que su pedido queda pendiente para su aprobación. \\ \hline
            Value: 10 \\ \hline
            Effort: 3 \\ \hline
        \end{tabular}
    \end{table}
    
    \strong{Justificación:} Funcionalidad escencial que debe tener la aplicación. \\
    
    \begin{table}[H]
        \centering
        \begin{tabular}{| p{1.5cm}| p{10.2cm} |}
            \hline
            \multicolumn{2}{|c|}{User Story: Ciudadano Resgistrado - Loguearse} \\
            \hline
            COMO & Ciudadano Registrado \\ \hline
            QUIERO & Ingresar a mi cuenta \\ \hline
            PARA & Acceder a las funcionalidades de usuario registrado \\ \hline
            \hline
        \end{tabular}
        \begin{tabular}{| p{12.118cm} |}
            \multicolumn{1}{|c|}{Tasks} \\
            \hline
            1. \\ \hline
            2. \\ \hline
            \hline
            \multicolumn{1}{|c|}{Criterios de Aceptación} \\
            \hline
            1. El usuario debe poder ingresar sus datos, para poder acceder al sistema. \\ \hline
            2. De ser correctos, ingresará al sistema registrado. \\ \hline
            3. De ser incorrectos, se le indicará esto y deberá ingresar los datos nuevamente. \\ \hline
            Value: 3 \\ \hline
            Effort: 5 \\ \hline
        \end{tabular}
    \end{table}
    
    \strong{Justificación:} Pensamos que estaría bueno hacer una distinción entre usuarios registrados y usuarios no registrados. El usuario registrado utiliza todas las funcionalidades, mientras que el que no está registrado sólo puede buscar bares. \\
    
    \begin{table}[H]
        \centering
        \begin{tabular}{| p{1.5cm}| p{10.2cm} |}
            \hline
            \multicolumn{2}{|c|}{User Story: Ciudadano Resgistrado - Calificar/Modificar Calificación} \\
            \hline
            COMO & Ciudadano Registrado \\ \hline
            QUIERO & Poder calificar las categorías o modificar las que ya realice de un bar seleccionado \\ \hline
            PARA & Aportar información del bar \\ \hline
            \hline
        \end{tabular}
        \begin{tabular}{| p{12.118cm} |}
            \multicolumn{1}{|c|}{Tasks} \\
            \hline
            1. \\ \hline
            2. \\ \hline
            \hline
            \multicolumn{1}{|c|}{Criterios de Aceptación} \\
            \hline
            1. El usuario puede ver las categorías del bar en una lista. Y (en caso de existir) las correspondientes calificaciones realizadas. \\ \hline
            2. El usuario puede calificar cada categoría que desee, con un puntaje de 1 a 5 estrellas. En caso de que ya exista una calificación, se modificara la misma. \\ \hline
            3. El usuario puede indicar si cancela la acción o si la confirma. \\ \hline
            4. Si la confirma, la información se sube a la base de datos  \\ \hline
            5. Sino, se descartan los cambios. \\ \hline
            Value: 8 \\ \hline
            Effort: 8 \\ \hline
        \end{tabular}
    \end{table}
    
    \strong{Justificación:} Nos pareció una buena idea que los usuarios puedan modificar sus calificaciones, para que de esta forma, si el bar mejora, esto se pueda ver reflejado en sus calificaciones. \\
    
    \begin{table}[H]
        \centering
        \begin{tabular}{| p{1.5cm}| p{10.2cm} |}
            \hline
            \multicolumn{2}{|c|}{User Story: Ciudadano Común - Registrar Usuario} \\
            \hline
            COMO & Ciudadano Común \\ \hline
            QUIERO & Registrarme en el sistema \\ \hline
            PARA & Acceder a funciones de usuario registrado (como agregar y calificar un bar) \\ \hline
            \hline
        \end{tabular}
        \begin{tabular}{| p{12.118cm} |}
            \multicolumn{1}{|c|}{Tasks} \\
            \hline
            1. \\ \hline
            2. \\ \hline
            \hline
            \multicolumn{1}{|c|}{Criterios de Aceptación} \\
            \hline
            1. El usuario debe tener la opción para registrarse al sistema. \\ \hline
            2. Elegida tal opción, el usuario deberá ingresar sus datos. \\ \hline
            3. El usuario podrá confirmar o descartar sus acciones. \\ \hline
            4. De confirmar (y de no haber ningún problema con los datos ingresados), se registrará al sistema e ingresará al mismo. \\ \hline
            5. En caso de haber problemas, se deberán corregir los errores marcados. \\ \hline
            6. Opcionalmente, se podrá registrar como adminsitrador con alguna condición especial (ejemplo: verificar un código especial). \\ \hline
            Value: 3 \\ \hline
            Effort: 8 \\ \hline
        \end{tabular}
    \end{table}
    
    \strong{Justificación:} Como decidimos hacer una diferencia entre usuarios registrados y usuarios no registrados, la funcionalidad de poder registrarse debe estar. \\
    
    \begin{table}[H]
        \centering
        \begin{tabular}{| p{1.5cm}| p{10.2cm} |}
            \hline
            \multicolumn{2}{|c|}{User Story: Ciudadano Común - Ver Ruta} \\
            \hline
            COMO & Ciudadano Común \\ \hline
            QUIERO & Ver la ruta más cercana de un bar seleccionado \\ \hline
            PARA & Llegar lo más rápido posible \\ \hline
            \hline
        \end{tabular}
        \begin{tabular}{| p{12.118cm} |}
            \multicolumn{1}{|c|}{Tasks} \\
            \hline
            1. \\ \hline
            2. \\ \hline
            \hline
            \multicolumn{1}{|c|}{Criterios de Aceptación} \\
            \hline
            1. Dado un bar seleccionado, y elegida la opción de visualización en el mapa, el usuario debe poder ver la ruta más cercana entre el bar seleccionado y el punto dado por el usuario. \\ \hline
            2. En todo momento el usuario puede volver a la pantalla anterior. \\ \hline
            Value: 9 \\ \hline
            Effort: 5 \\ \hline
        \end{tabular}
    \end{table}
    
    \strong{Justificación:} Esta es una funcionalidad escencial que debe tener la aplicación. \\

\begin{itemize}
    
    \item User Story: Admin - Aprobar Bar
    
    COMO administrador QUIERO evaluar la aprobación de un bar PARA moderar el contenido de la aplicación.

    \strong{Justificación:} Luego de que un usuario registrado agregue un bar, un admin debe moderar el contenido, para que se esta forma, la aplicación no se llene de información falsa.
    
    \strong{Criterios de Aceptación:}
    
    \strong{Value:}
    
    \strong{Effort:}
    
    \item User Story: Admin - Modificar/Eliminar un Bar
    
    COMO Admin QUIERO poder modificar la información de un bar seleccionado e incluso borrarlo PARA poder moderar la información
    
    \strong{Justificación:} Queremos que la posibilidad de modificar/eliminar un bar este disponible para el administrador, ya que podría ser útil si se ingreso información falsa, o si se desea quitar un bar de la app
    
    \strong{Criterios de Aceptación:}
    
    \strong{Value:}
    
    \strong{Effort:}
    
    \item User Story: Ciudadano Registrado - Agregar Bar
    
    COMO ciudadano registrado en la aplicación QUIERO poder agregar un nuevo bar PARA que aparezca en la aplicación
    
    \strong{Justificación:} Funcionalidad esencial que debe tener la aplicación.

    \item User Story: Ciudadano Registrado - Logearse
    
    COMO Ciudadano Registrado QUIERO ingresar a mi cuenta PARA acceder a las funcionalidades de usuario registrado
    
    \strong{Justificación:} Pensamos que estaría bueno hacer una distinción entre usuarios registrados y usuarios no registrados. El usuario registrado utilizar todas las funcionalidades, mienstras que el que no esta registrado solo puede buscar bares.
    
    \item User Story: Ciudadano Registrado - Calificar/Modificar Calificación
    
    COMO ciudadano registrado QUIERO poder calificar las categorías o modificar las que ya realice de un bar seleccionado PARA aportar información del bar.
    
    \strong{Justificación:} Nos pareció una buena idea que los usuarios puedan modificar sus calificaciones, para que de esta forma, si el bar mejora, esto se pueda ver reflejado en sus calificaciones
    
    \item User Story: Ciudadano Común - Registrar Usuario
    
    COMO Ciudadano Común QUIERO registrarme en el sistema PARA acceder a funciones de usuario registrado (como agregar y calificar un bar)
    
    \strong{Justificación:} Como decidimos hacer una diferencia entre usuarios registrados y usuarios no registrados, la funcionalidad de poder registrarse debe estar.
    
    \item User Story: Ciudadano Común - Ver Ruta
    
    COMO Ciudadano Común QUIERO ver la ruta mas cercana de un bar seleccionado PARA llegar lo mas rápido posible
    
    \strong{Justificación:} Esta es una funcionalidad escencial que debe tener la aplicacion
    
    
\end{itemize}
